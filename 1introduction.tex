\chapter{Introduction}
\label{chapter:intro}

\section{Background and Objectives}
After the web turning from 1.0 to 2.0 where the network as platform, spanning all connected devices. Unlike Web 1.0, applications are treated as continually-updated services that get better the more people use it. This is done by consuming and fusing data from multiple sources, including individual users, while providing their own data and services in a form that allows fusing by third parties \cite{o2005web}.

Web browsers are now commonly installed in smart electronic equipments such as computers, smart phones, and tablets which can connect to the Internet. Typical contents, presented by web browsers, are text and multimedia such as images, audio and video. These contents are stored in other computers connected by the Internet. Our work started at this point: a browser directly communicates with an Internet-enabled device. To make the work interesting, devices should have good interoperability i.e., they should be able to communicate with sensors and actuators, or even directly with other devices.

Usability is another important element of Web 2.0. Web pages are not just pages that have simple colours, logos, and animations, but they are more like desktop applications. Thus, web pages are turning to web applications  \cite{Lewis:2006:WEB:1217666.1217669}.

With the radical improvement of content usability, Web applications contents have been presenting more dynamic. Data can be retrieved from multiple sources in real time and assembled on a single web application or page \cite{Lewis:2006:WEB:1217666.1217669}. A web browser, one of the user agents that retrieves contents identified by Uniform Resource Identifier (URI) \cite{ArchitectWorldWideWebVolumeOne}, is the gateway for users to the Internet.

With the purpose of presenting rich content to users, a web browser communicates with a web server to fetch the data. The interaction between a browser and a web server are, traditionally, based on HTTP requests. That is, a browser sends HTTP requests to a web server, specifying the desired data; the server, replies to the browser in HTTP responses with the data, if available. At the application level on the browser side, an HTTP request is usually sent by using techniques, such as, polling or long-polling over XMLHttpRequest (or Ajax). When using Ajax, data are retrieved from a remote server by sending an HTTP request asynchronously; for more other data, the web browser sends other HTTP requests. In this thesis, we address scalability and performance of web applications involving Internet enabled devices. Specifically, Ajax increases the frequency of requesting the web server, with repeatedly connections and disconnections; or rises the workload of the web server to keep the requests open. The problem becomes severe in the scenario of Internet of Things (IoT) with a large amount of interconnected devices.

However, this approach can radically be improved by removing the redundancy of HTTP requests. Then we transfer all data over a single TCP connection, which is what we called WebSockets. The goal of this technology is to provide a mechanism for browser-based applications that need two-way communication with servers without relying on opening multiple HTTP connections \cite{rfc64552012web}.

In effort to solve the Ajax abuse, bidirectional and real-time communication problems in IoT, we provide a solution using WebSockets with its subprotocol, the WebSocket Application Messaging Protocol (WAMP). In the thesis, we simulated and visualised the behaviours of IoT devices over an HTML5 application using publish/subscribe (Pub/Sub) and Remote Procedure Call (RPC) pattens, providing by WAMP. These two pattens cover the two most important interaction in IoT: notification and control. Additionally, we present another pattern, the database-centric approach, for web-based IoT applications. 

\section{Contributions}

In the thesis, we defined different patterns for web-based IoT applications suitable for time-sensitive and low overhead communications. Amongst, we focused on Publish/Subscribe (Pub/Sub), Remote Procedure Call (RPC) and database-centric approach, as they cover the most desired features that the web-based IoT applications demand. Furthermore, we implemented the web-based IoT applications in two relevant scenarios, namely, home automation and social sensing. Specifically, we combined the first two patterns (Pub/Sub and RPC) with one HTML5 web technology, the WebSocket, and applied them to a home automation scenario, enabling the web-based IoT application cross-platforms (such as mobiles and desktops). For the database-centric approach, we presented this pattern in a social sensing scenario, ensuring that the whole web-based IoT application is scalable, as the IoT data may increase very quickly.

For the purpose of ensuring realtime communication in the web-based IoT applications, whilst reserving scalability and security, we developed a scalable and robust architecture for each scenario of the aforementioned applications, based on the corresponding requirement. Through the architectures, we managed to achieve scalability and security in our implementation result. Moreover, we presented a way to visualise IoT data. This is an essential aspect to improve usability for the end users to inspect, review and manage IoT data.

We finally compared and evaluated different technologies and protocols for IoT-based communications. Specifically, we evaluated the performance of potential web technologies and protocols. This comparison efficiently highlights the protocols that are not suitable for web-based IoT applications. For the home automation scenario, we further focused on comparison message overhead and Roundtrip Transfer Time (RTT). The obtained results show that our approach has a better performance than other approaches commonly used in the IoT.

\section{Structure}

The rest of the thesis is organised as follows. Chapter 2 introduces the background about web technologies, including the WebSocket and its subprotocols. Chapter 3 discusses the background about IoT and its most relevant aspects; it will also overview relevant Internet-based communication protocols. Chapter 4 presents the design and implementation of the web-based IoT applications, while Chapter 5 provides performance evaluation of protocols suitable to IoT applications. Finally, Chapter 6 concludes the thesis. 