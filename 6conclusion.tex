\chapter{Future and Conclusion}
\label{chapter:future-and-conclusion}

At the current stage, the implementation is done at software level: hardware part are simulated. Therefore, the development process can be trimmed to fit in the scope of the thesis. However, we, carefully, preserved the essential features of each hardware component and simulated at the software level. This means, there should not be many different behaviours for further deployment on hardware components. 

One example, which could link all the features back to hardware, is Raspberry Pi\footnote{http://www.raspberrypi.org/}. Raspberry Pi supports Python as its official programming language by default, but any language which will compile for ARMv6 can be used. This implies that NodeJS can be ported on Raspberry Pi. Moreover, the embedded web sever and its upper layers, which we implemented, are not restricted by NodeJS, meaning the architecture can be built with other programming languages, such as Python. A light, a switch and a sensor can, subsequently, be connected to Raspberry Pi. We can, furthermore, use a phone's sensors as a data source, and transmit the data from the phone to Raspberry Pi.

The thesis is about applying web technology on Internet of Things. We focused on the WebSocket protocol as the part of web technology. The WebSocket creates a bi-directional connection between a server and a web browser. We highlighted its advantage by comparing it with the traditional method, HTTP, according to the behaviours of fetching resources from a web server. Furthermore, we chose WAMP as a subprotocol to support WebSocket complementarily, enabling WebSocket unprecedentedly functionality to interact with Internet of Things services. 

Moving onto the part of Internet of Things, We integrated a sensor (light sensor) and an actuator (light switch/dimmer) into the application, blending with the web technology. The major feature of the sensor (sending data) and that of the actuator (commanding and configuration) perfectly match the massage patterns, Pub/Sub and RPC, which WAMP provides. Consequently, the smart lighting and controlling Internet of Things service emerged. 

Additionally, we compared the smart lighting and controlling application with social network IoT sensing application, an application interacting with IoT services with traditional method (HTTP), from the perspective of architectures, implementation, scalability and security. 

Moreover, We compared WAMP with other registered WebSocket subprotocols: MBWS, SOAP and STOMP from the viewpoint of features; and with other potential protocols: CoAP and MQTT, from the viewpoint of practicability. We further evaluated WAMP and MQTT with communication overheads and RTT. 

There are, surely, many potential room for the future research on the blending of web technology with Internet of Things. The upsurge in bleeding-edge web technology might be supportive to the future Internet of Things services research, e.g., for the efficiency, maintainability and scalability. 