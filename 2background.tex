\chapter{Background}
\label{chapter:background} 

The WebSocket Protocol enables full-duplex communication through a single socket between a client and a remote server. For this reason, this technology decreases the amount of opening ports on server sides, comparing with the traditional mean of retrieving resource. Moreover, this protocol is not just an enhancement of the current HTTP protocol; it, however, represents enormous advance, especially for real-time, event-driven communication \cite{lubbers2010html5}.

\section{Traditional Communication}

Usually, the client/server model is designed as a request/response model, where the server does not actively communicate with clients. The client/server model is one of the network architectures to connect computers or applications. A client could be a computer, or an application, such as a web browser. 

A server is, usually, a (powerful) computer that keeps some applications running as constantly as possible. Those applications could be a database, a web server (application), or a printer server (application) that links to a physical printer. 

The HTTP protocol is a request/response protocol \cite{fielding1999hypertext}, where a client, or rather a user agent (UA) communicates with a server through a request and a response message(s). In other words, a client retrieves resources from a server in a request message, while the server hands over the contents in a response message. This is a general model that a client retrieves data from a server. 

Generally, there are three ways to retrieve data \cite{lubbers2010html5}: polling, long polling and streaming, while all of these are uni-directional. 

With polling, a client sends a request to a server, e.g. a web browser sends a request with Asynchronous JavaScript and XML (Ajax) \cite{garrett2005ajax} to a web server and expects data. This is a solution of high quality, if the client knows exactly the time when the data is available. However, as you might notice, the server might not have the contents when the client requests, or the contents are temporarily not available. If this situation would happen in practise, the client would have repeated polling for data. This will result in a large redundant headers generated by HTTP protocol. Consequently, the approach leads a heavy workload to the server and will not be suitable in the scenario, such as, a sensor network having thousands of sensors. 

Another method, Long polling, is an improved approach, with respect to the frequency of request. Compared with polling, long polling does similarly, expect that the server holds the client's request for a certain period until timeout. Within this period, the server will send a response(s) back to the client, if there is data available. This method reduces the frequency of HTTP requests, but it increases the workload of a server, such as, ports handling. Generally, an HTTP request is initiated over TCP/IP connections (but any other protocols that guarantee a reliable transport can be replaced) \cite {fielding1999hypertext}. In TCP/IP connections, a port number is 16 bits \cite{postel2003rfc}, which means the maximum number of ports is \(  2^16  = 65536 \). However, one characteristic in sensor network is scalability, which means the number of sensor nodes could yield to a larger amount than that of the ports. Even though all ports are dedicated to communication, it is insufficient to support a enormous sensor network. 

As for the HTTP streaming, or known as HTTP server push, a server sends a response and keeps the connection open and alive, where the connection is triggered by a client's request. The drawback of this approach is that streaming is still encapsulated in HTTP, and thus buffer mechanism might increase the latency of messages delivering. Moreover, buffer is not an approach for an IoT device, considering the device's energy constraint.

\section{The Internet of Things and WebSocket}

In the concept of Internet of Things (IoT), the networked objects, called devices, are deployed worldwide and connected over the internet. Moreover, the IoT device is individually addressable, so that each device is interconnected and can be accessed through the standards of the web. With the purpose of performing a better communication with IoT devices, the WebSocket protocol enables two-way communication and is arranged to substitute the disadvantage that existing in the traditional client/server model.

The WebSocket protocol supports data frame including text and binary. Once the WebSocket connection has been successfully established, generally, data frames transfer between a client and a server. However, in consideration of optimising WebSocket connections between a client and a server, the client can request that the server use a specific subprotocol by including the |Sec-WebSocket-Protocol| filed in its handshakes\cite{rfc64552012web}. A subprotocol is a application-level protocols layered over the WebSocket protocol\cite{rfc64552012web}. A subprotocol could be implemented by the servers and can be also versioned. The server chooses one or none subprotocol. 

\section{WAMP}

WebSocket Application Messaging Protocol (WAMP)\footnote{http://wamp.ws/} is a subprotocol officially registered in WebSocket Protocol Registries \footnote{http://www.iana.org/assignments/websocket/websocket.xml\#subprotocol-name} , which gives structure to messaging by providing two hight level messaging patterns: Publish/Subscribe (Pub/Sub) and Remote Procedure Calls (RPC).

WAMP fills in the blanks in the WebSocket protocol: a low-level specification which originally provides raw messaging communication (text or binary). Moreover, WAMP does not make any extra standard, but it bases on established Web standards: WebSocket, JSON and URI. The WebSocket standard is the default transport channel where WAMP is binding. With the WebSocket Standard, WAMP is assumed to be given a reliable, ordered, and full-duplex message channel. 

\subsection{JSON}

JavaScript Object Notation (JSON) is a text format for the serialisation of structured data. It is derived from the object literals of JavaScript, as defined in the ECMAScript Programming Language Standard, Third Edition [ECMA] \cite{crockford2006application}. JSON has numbers, strings, booleans and null as its primitive types. Additionally, it has objects and arrays as structured types. JSON becomes popular to exchange data between in web services or applications. With JSON, WAMP message payload is serialisable and provides data types according to those of JSON. 

\subsection{URI}

Uniform Resource Identifiers (URIs) provide a simple and extensible means for identifying a resource. URIs have a global scope and are interpreted consistently regardless of context \cite{masinter2005uniform}. WAMP binds to URIs by default, then it assumes global assignments and resolution to be unique. URIs are used as IDs for both topics in Pub/Sub and procedures in RPC. 

\section{Other Registered WebSocket Subprotocols}

Apart from WAMP, there are MessageBroker WebSocket Subprotocol (MBWS), Simple Object Access Protocol(SOAP) and Simple Text Oriented Messaging Protocol (STOMP) can be used as subprotocols, according to the WebSocket Protocol Registries.

MBWS is used by messaging clients to send messages to, and received messages from, an Internet message broker. The Internet message broker is also called a message broker which queues messages, sent by clients, for asynchronous deliveries to clients. \cite{hapner2012messagebroker}. 

Like WAMP, MBWS provides full-duplex and reliable transport; Unlike WAMP, MBWS does not obviously provide higher level messaging patterns. MBWS, however, defines a binary message frame starting with an MBWS binary header, and a text message frame starting with an MBWS text header. MBWS provides two additional features. Firstly, it supports connection recovery, which has not defined by the WebSocket protocol. Secondly, it supports metadata in MBWS message headers. MBWS has a light weight version, called MessageBrokerLight WebSocket Subprotocol (MBLWS) which only supports Message Metadata \cite{hapner2012messagebroker}. However, MBWS is still a draft. 

SOAP is a lightweight protocol intending for exchanging structured information in a decentralised and distributed environment. SOAP uses XML technology to exchange data. \cite{gudginsoap} Microsoft, additionally, defines SOAP Over WebSocket Protocol Binding (MS-SWSB) \footnote{http://msdn.microsoft.com/en-us/library/hh536812.aspx}, a subprotocol for the WebSocket Protocol. The subprotocol involves a Web Services Description Language (WSDL) and supports message exchange patterns (MEPs). With MEPs, SOAP supports one-way, two-way and request/response message exchange patterns, which involves Pub/Sub and RPC.

Simple (or Streaming) Text Orientated Messaging Protocol (STOMP) \footnote{http://stomp.github.io} is a frame based protocol. It refers to a STOMP frame includes a command, optional headers and a body. STOMP is text based with UTF-8 encoding by default, but it, optionally, support binary messages. As for message patterns, STOMP version 1.2 supports Pub/Sub. An example of Pub/Sub model using STOMP with WebSocket is discussed in \cite{wang2012definitive}.

Table \ref{table:subprotocol-comparison} is a comparison summary among WAMP, MBWS, SOAP and STOMP:

% If you need to have linefeeds (\\) inside a cell, you must create a new
% paragraph-formatting environment inside the cell. Most common ones are 
% the minipage-environment and the \parbox command (see LaTeX documentation
% for details; or just google for ``LaTeX minipage'' and ``LaTeX parbox'').
\begin{table}
\begin{tabular}{|p{2.2cm}|>{\centering\arraybackslash}p{2.2cm}|>{\centering\arraybackslash}p{3cm}|>{\centering\arraybackslash}p{1.5cm}|>{\centering\arraybackslash}p{2.9cm}|} 
% Alignment of sells: l=left, c=center, r=right. 
% If you want wrapping lines, use p{width} exact cell widths.
% If you want vertical lines between columns, write | above between the letters
% Horizontal lines are generated with the \hline command:
\hline % The line on top of the table
\textbf{ } & \textbf{WAMP} & \textbf{MBWS} & \textbf{SOAP} & \textbf{STOMP} \\ 
\hline 
% Place a & between the columns
% In the end of the line, use two backslashes \\ to break the line,
% then place a \hline to make a horizontal line below the row 
\textbf{Data Representation} & JSON & binary and text with MBWS headers & XML & binary and text with headers \\ 
\hline
\textbf{Message Patten} & Pub/Sub, RPC & not specified & MEP & Pub/Sub \\
% The multicolumn command takes the following 3 arguments: 
% the number of cells to merge, the cell formatting for the new cell, and the
% contents of the cell
\hline
\textbf{Connection Recovery} & no & yes & no & no \\
\hline
\textbf{Metadata} & currently no & yes & yes & no \\
\hline
\textbf{Released} & yes & draft & yes & yes \\
\hline
\end{tabular} % for really simple tables, you can just use tabular
% You can place the caption either below (like here) or above the table
\caption{WAMP and other WebSocket subprotocols}
\label{table:subprotocol-comparison}
\end{table} % table makes a floating object with a title
